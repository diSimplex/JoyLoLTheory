% !TEX root = definitions.tex
% !LPiL preamble = ../jThPreamble.tex
% !LPiL postamble = ../jThPostamble.tex


% compare the interval structure of:
% - martinPanagaden2006spacetimeTalk
% - martinPanangaden2006spacetimeIntervalsRelativity
% - martinPanagaden2010domansGenRelativity


\section*{Definitions}

We generalize Winskel's definition of Event Structures
\cite{winskel1980phdThEvents, winskel1986eventStruct, winskelNielsen1995concurrency,
rideauWinskel2011concurrentStrategies} (and countless other papers).

% see also:
% Micheal Shield's book

\begin{definition}
Consider a partially ordered set, \tuple{E, \leq}. We call the elements of the set, $E$, `events'.
We call the partial order, a `causal order'.

A partially ordered set is locally finite if:

 $$e_1, e_2 \in E \text{ and } e_2 \leq e_1 \implies \set{e' \in E \suchThat e_2 \leq
   e' \leq e_1} \text{is finite}.$$

A subset, $X$, of $E$ is (with respect to the causal order)
\begin{itemize}
  \item \define{dwnCls}{downward-closed} : $e \in X, e' \in E, e' \leq e
    \implies e' \in X$,
  \item \define{upward-closed}{upward-closed} : $e \in X, e' \in E, e \leq e'
    \implies e' \in X$,
  \item \define{convex}{convex} : $e_1, e_2 \in X, e' \in E, e_1 \leq e' \leq e_2
    \implies e' \in X$.
\end{itemize}

An \define{evStruct}{Event Structure}, \tuple{E, \leq, Con}, consists of a
\emph{pre-event structure} together with a and a non-empty
\define{consitency}{Consistency relation}, Con, consisting of finite subsets of
$E$, for which:

\begin{itemize}
  \item 
  \item $\set{e} \in Con$ for all $e \in E$,
  \item $Y \subseteq X \in Con \implies Y \in Con$, and
    \item $X \in Con$ and $\forall e_1, e_2 \in X, e_2 \leq e' \leq e_1 \implies X
  \union \set{e'} \in Con$.
\end{itemize}
\end{definition}

This is essentially Winskel's definition except that we replace his use of
``downward closed'' intervals by ``convex'' intervals.

\begin{definition}
The (finite) configurations, \config{E}, of an event structure, $E$, consists of
those finite subsets $x \subseteq E$ for which
\begin{itemize}
  \item \define{consistent}{Consistent}: $x \in Con$,
  \item \define{convex}{Convex}: $\forall e_1, e_2 \in x, e' \in E, e_2 \leq e'
    \leq e_1 \implies e' \in x$
\end{itemize}
\end{definition}

Again, in this definition, we replace Winskel's use of ``downward-closed'' by
``convex''.

\begin{definition}
A \define{stableFam}{Stable Family} is a non-empty family, \stbFam{F}, of finite
subsect of $E$, called \define{configurations}{configurations}, for which
\begin{itemize}
  \item \define{completeness}{Completeness}: $\forall Z \subseteq \stbFam{F}, Z
    \uparrow \implies \bigcup Z \in \stbFam{F}$,
  \item \define{coincidentFree}{Coincidence-freeness} : For all $x \in
    \stbFam{F}, e, e' \in x$ with $e \ne e'$, 
    $$ \exists y \in \stbFam{F}, y \subseteq x \& (e \in y \implies e' \notin y)$$,
  \item \define{stability}{Stability} : $\forall Z \subseteq \stbFam{F}, Z \ne
    \emptyset \& Z \uparrow \implies \bigcap Z \in \stbFam{F}$
\end{itemize}
where we define $Z \uparrow$ to mean $\exists x \in \stbFam{F}, \forall z \in Z, z \subseteq x$ 
\end{definition}

This \emph{is} Winskel's original definition.